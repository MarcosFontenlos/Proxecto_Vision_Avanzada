\documentclass{article}
\usepackage[galician]{babel}
\usepackage[utf8]{inputenc}
\usepackage{amsmath}
\usepackage{amsfonts}
\usepackage{esint} 				%simbolos de integrales
\usepackage{amssymb}
\usepackage{graphicx,fancybox}
\usepackage{mathrsfs} 		

\title{Comportamento baseado en visión para Robulab 10}
\author{Marcos Fontenlos Ferreiro, Alejandro Solar Iglesias}
\date{\today}

\begin{document}

\maketitle

\section*{Proposta de traballo}

\section{Extracto}

Os obxectivos son implementar un sistema de visión artificial baseado en redes neuronais para a detección de persoas no robot Robulab 10.
O sistema deberá ser capaz de detectar persoas en tempo real e gobernar o robot para que siga o seu movemento. Tamén intentaremos implementar un sistema 
de detección gestual empregando a librería \textit{mediapipe} facendo que o robot siga algunha consigna particular para cada xesto. Simultáneamente á implementación 
trataremos de documentar o proceso con información detallada para que quede un manual de uso para próximos usuarios.

\section{Introdución e contextualización}

\subsection{Visión xeral}

É importante a aplicación das técnicas que vemos no curso de Visión Artificial Avanzada en plataformas hardware reais e non só en sistemas encaixados básicos coma un PC.
Os robots son nunha última instancia o noso obxeto de estudo e é importante compaxinar os coñecementos adquiridos en diversas asignaturas para conseguir 
comportamentos complexos e útiles nos robots. 

\subsection{Antecedentes do tema}

Evidentemente a detección de persoas e xestos non é un tema novedoso e a nosa intención non é revolucionar estes campos, senón aplicar as técnicas máis prometedoras 
en cadanseu ámbito na plataforma robótica. Maior é a preocupación polo manexo do propio robot que polo propio sistema de visión; non existe moita variedade de proxectos nesta plataforma 
ou polo menos non son de fácil acceso máis aló da documentación oficial.

\subsection{Relevancia/Impacto}

De novo, o proxecto non vai traer unha gran revolución na robótica nin na visión artificial máis aló da utilidade dun robot sigue persoas e detector de xestos, pero si que deixará unha documentación que poderá ser útil para futuros proxectos.
Sobre todo para o traballo con esta plataforma hardware concreta. 
\section{Metodoloxía proposta}
Empregaremos a librería \textit{mediapipe} e \texttt{Yolo} para a detección de persoas e xestos xunto ca documentación oficial de Robulab 10 para a integración do sistema 
de recoñecemento no robot.

\section{Plan de traballo}
\begin{itemize}
    \item Aproximación á librería \textit{mediapipe} e \texttt{Yolo} para a detección de persoas e xestos. Esta última xa a temos traballada na asignatura de Visión Artificial Avanzada.
    \item Estudo do funcionamento do robot Robulab 10 e a súa integración coas librerías de visión artificial.
    \item Implementación do sistema de detección de persoas e xestos.
    \item Probas de funcionamento do sistema de visión e do robot.
    \item Documentación do proceso de implementación e funcionamento do sistema de visión. (De forma máis ou menos simultánea ás dúas anteriores)
    \item Posible implementación de xestos personalizados para o control do robot.
\end{itemize}

\end{document}
